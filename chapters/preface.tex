\chapter{Preface}

\section{What is this book about?}

This is a book introducing visual SLAM, and it is probably the first Chinese book solely focused on this specific topic.

So, what is SLAM?

SLAM stands for \textbf{S}imultaneous \textbf{L}ocalization \textbf{a}nd \textbf{M}apping. It usually refers to a robot or a moving rigid body, equipped with a specific \textbf{sensor}, estimates its own \textbf{motion} and builds a \textbf{model} (certain kinds of description) of the surrounding environment, without a \textit{priori} information\cite{Davison2007}. If the sensor referred here is mainly a camera, it is called \textbf{Visual SLAM}.

Visual SLAM is the subject of this book. We deliberately put a long definition into one single sentence, so that the readers can have a clear concept. First of all, SLAM aims at solving the \textit{positioning} and \textit{map building} issues at the same time. In other words, it is a problem of how to estimate the location of a sensor itself, while estimating the model of the environment. So how to achieve it? This requires a good understanding of sensor information. A sensor can observe the external world in a certain form, but the specific approaches for utilizing such observations are usually different. And, why is this problem worth spending an entire book to discuss? Simply because it is difficult, especially if we want to do SLAM in \textbf{real time} and \textbf{without any a priory knowledge}. When we talk about visual SLAM, we need to estimate the trajectory and map based on a set of continuous images (which form a video).

This seems to be quite intuitive. When we human beings enter an unfamiliar environment, aren't we doing exactly the same thing? So, the question is whether we can write programs and make computers do so. 

At the birth of computer vision, people imagined that one day computers could act like humans, watching and observing the world, and understanding the surrounding environment. The ability to explore unknown areas is a wonderful and romantic dream, attracting numerous researchers striving on this problem day and night~\cite{Hartley2003}. We thought that this would not be that difficult, but the progress turned out to be not as smooth as expected. Flowers, trees, insects, birds, and animals, are recorded so differently in computers: they are simply matrices consisted of numbers. To make computers understand the contents of images is as difficult as making us humans understand those blocks of numbers. We didn't even know how we understand images, nor do we know how to make computers do so. However, after decades of struggling,  we finally started to see signs of success - through Artificial Intelligence (AI) and Machine Learning (ML) technologies, which gradually enable computers to identify objects, faces, voices, texts, although in a way (probabilistic modeling) that is still so different from us. On the other hand, after nearly three decades of development in SLAM, our cameras begin to capture their movements and know their positions, although there is still a huge gap between the capability of computers and humans. Researchers have successfully built a variety of real-time SLAM systems. Some of them can efficiently track their locations, and others can even do three-dimensional reconstruction in real-time.

This is really difficult, but we have made remarkable progress. What's more exciting is that, in recent years, we have seen the emergence of a large number of SLAM-related applications. The sensor location could be very useful in many areas: indoor sweeping machines and mobile robots, self-driving cars, Unmanned Aerial Vehicles (UAVs), Virtual Reality (VR), and Augmented Reality (AR). SLAM is so important. Without it, the sweeping machine cannot maneuver in a room autonomously, but wandering blindly instead; domestic robots can not follow instructions to reach a certain room accurately; Virtual Reality will always be limited within a prepared space. If none of these innovations could be seen in real life, what a pity it would be.

Today's researchers and developers are increasingly aware of the importance of SLAM technology. SLAM has over 30 years of research history, and it has been a hot topic in both robotics and computer vision communities. Since the 21st century, visual SLAM technology has undergone a significant change and breakthrough in both theory and practice and is gradually moving from laboratories into the real-world. At the same time, we regretfully find that, at least in the Chinese language, SLAM-related papers and books are still very scarce, making many beginners of this area unable to get started smoothly. Although the theoretical framework of SLAM has basically become mature, to implement a complete SLAM system is still very challenging and requires a high level of technical expertise. Researchers new to the area have to spend a long time learning a significant amount of scattered knowledge and often have to go through several detours to get close to the real core.

This book systematically explains the visual SLAM technology. We hope that it will (at least in part) fill the current gap. We will detail SLAM's theoretical background, system architecture, and the various mainstream modules. At the same time, we place great emphasis on practice: all the important algorithms introduced in this book will be provided with runnable code that can be tested by yourself so that readers can reach a deeper understanding. Visual SLAM, after all, is a technology for real-applications. Although the mathematical theory can be beautiful, if you are not able to convert it into lines of code, it will be like a castle in the air, which brings little practical impact. We believe that practice verifies true knowledge, and practice tests true passion. Only after getting your hands dirty with the algorithms, you can truly understand SLAM and claim that you have fallen in love with SLAM research.

Since its inception in 1986~\cite{Smith1986}, SLAM has been a hot research topic in robotics. It is very difficult to give a complete introduction to all the algorithms and their variants in the SLAM history, and we consider it as unnecessary as well. This book will first introduce the background knowledge, such as projective geometry, computer vision, state estimation theory, Lie Group and Lie algebra, etc. On top of that, we will be showing the trunk of the SLAM tree, and omitting those complicated and oddly-shaped leaves. We think this is effective. If the reader can master the essence of the trunk, they have already gained the ability to explore the details of the research frontier. So we aim to help SLAM beginners quickly grow into qualified researchers and developers. On the other hand, even if you are already an experienced SLAM researcher, this book may still reveal areas that you are unfamiliar with, and may provide you with new insights.

There have already been a few SLAM-related books around, such as ``Probabilistic Robotics''~\cite{Thrun2005}, ``Multiple View Geometry in Computer Vision''~\cite{Hartley2003}, ``State Estimation for Robotics: A Matrix-Lie-Group Approach''\cite{Barfoot2016}, etc. They provide rich content, comprehensive discussions and rigorous derivations, and therefore are the most popular textbooks among SLAM researchers. However,  there are two important issues: Firstly, the purpose of these books is often to introduce the fundamental mathematical theory, with SLAM being only one of its applications. Therefore, they cannot be considered as specifically visual SLAM focused. Secondly, they place great emphasis on mathematical theory but are relatively weak in programming. This makes readers still fumbling when trying to apply the knowledge they learn from the books. Our belief is: only after coding, debugging, and tweaking algorithms and parameters with his own hands, one can claim a real understanding of a problem.

In this book, we will be introducing the history, theory, algorithms, and research status in SLAM, and explaining a complete SLAM system by decomposing it into several modules: \textit{visual odometry}, \textit{back-end optimization}, \textit{map building}, and \textit{loop closure detection}. We will be accompanying the readers' step by step to implement the core algorithms of each module, explore why they are effective, under what situations they are ill-conditioned, and guide them through running the code on their machines. You will be exposed to the critical mathematical theory and programming knowledge and will use various libraries including \textit{Eigen}, \textit{OpenCV}, \textit{PCL}, \textit{g2o}, and \textit{Ceres}, and master their use in the Linux operating system.

Well, enough talking, wish you a pleasant journey!

\section{How to use this book?}

This book is entitled ``14 Lectures on Visual SLAM''. As the name suggests, we will organize the contents into ``lectures'' like we are learning in a classroom. Each lecture focuses on one specific topic, organized in a logical order. Each chapter will include both a theoretical part and a practical part, with the theoretical usually coming first. We will introduce the mathematics essential to understand the algorithms, and most of the time in a narrative way, rather than in a ``definition, theorem, inference'' approach adopted by most mathematical textbooks. We think this will be much easier to understand, but of course with the price of being less rigorous sometimes. In practical parts, we will provide code and discuss the meaning of the various parts, and demonstrate some experimental results. So, when you see chapters with the word ``practice'' in the title, you should turn on your computer and start to program with us, joyfully.

The book can be divided into two parts: The first part will be mainly focused on fundamental math knowledge, which contains:
\begin{enumerate}
	\item Lecture 1: preface (the one you are reading now), introducing the contents and structure of the book.
	\item Lecture 2: an overview of a SLAM system. It describes each module of a SLAM system and explains what they do and how they do it. The practice section introduces basic C++ programming in a Linux environment and the use of an IDE\@.
	\item Lecture 3: rigid body motion in 3D space. You will learn knowledge about rotation matrices, quaternions, Euler angles, and practice them with the Eigen library.
	\item Lecture 4: Lie group and Lie algebra. It doesn't matter if you have never heard of them. You will learn the basics of Lie group, and manipulate them with Sophus.
	\item Lecture 5: pinhole camera model and image expression in computer. You will use OpenCV to retrieve the camera's intrinsic and extrinsic parameters, and then generate a point cloud using the depth information through PCL (Point Cloud Library).
	\item Lecture 6: nonlinear optimization, including state estimation, least squares, and gradient descent methods, e.g.\ Gauss-Newton and Levenburg-Marquardt. You will solve a curve-fitting problem using the Ceres and g2o library.
	
	From lecture 7, we will be discussing SLAM algorithms, starting with Visual Odometry (VO) and followed by the map building problems: 
	
	\item Lecture 7: feature-based visual odometry, which is currently the mainstream in VO. Contents include feature extraction and matching, epipolar geometry calculation, Perspective-n-Point (PnP) algorithm, Iterative Closest Point (ICP) algorithm, and Bundle Adjustment (BA), etc. You will run these algorithms either by calling OpenCV functions or by constructing your own optimization problem in Ceres and g2o.
	
	\item Lecture 8: direct (or intensity-based) method for VO. You will learn the principle of optical flow and direct method, and then use g2o to achieve a simple RGB-D direct method based VO (the optimization in most direct VO algorithms will be more complicated).
	
	\item Lecture 9: back-end optimization. We will discuss Bundle Adjustment in detail, and show the relationship between its sparse structure and the corresponding graph model. You will use Ceres and g2o separately to solve the same BA problem.
	
	\item Lecture 10: pose graph in the back-end optimization. Pose graph is a more compact representation for BA which marginalizes all map points into constraints between keyframes. You will use g2o and gtsam to optimize a pose graph.
	
	\item Lecture 11: loop closure detection, mainly Bag-of-Word (BoW) based method. You will use dbow3 to train a dictionary from images and detect loops in videos. 
	
	\item Lecture 12: map building. We will discuss how to estimate the depth of feature points in monocular SLAM  (and show why they are unreliable). Compared with monocular depth estimation, building a dense map with RGB-D cameras is much easier. You will write programs for epipolar line search and patch matching to estimate depth from monocular images, and then build a point cloud map and octagonal treemap from RGB-D data.
	
	\item Lecture 13: a practice chapter for VO. You will build a visual odometer framework by yourself by integrating the previously learned knowledge and solve problems such as frame and map point management, keyframe selection, and optimization control.
	
	\item Lecture 14: current open-source SLAM projects and future development direction. We believe that after reading the previous chapters, you will be able to understand other people's approaches easily and be capable to achieve new ideas of your own.
\end{enumerate}

Finally, if you don't understand what we are talking about at all, congratulations! This book is right for you! 

\section{Source code}

All source code in this book is hosted on Github:

{\hfill\url{https://github.com/gaoxiang12/slambook2}\hfill}

Note the slambook2 refers to the second version in which we added a lot of extra experiments.


It is strongly recommended that readers download them for viewing at any time. The code is divided by chapters, for example, the contents of the 7th lecture will be placed in folder ``ch7''. In addition, some of the small libraries used in the book can be found in the ``3rd party'' folder as compressed packages. For large and medium-sized libraries like OpenCV, we will introduce their installation methods when they first appear. If you have any questions regarding the code, click the ``Issues'' button on GitHub to submit. If there is indeed a problem with the code, we will make changes in a timely manner. Even if your understanding is biased, we will still reply as much as possible. If you are not accustomed to using Git, you can also click the button on the right which contains the word ``download'' to download a zipped file to your local drive.

\section{Oriented readers}

This book is for students and researchers interested in SLAM. Reading this book needs certain prerequisites, we assume that you have the following knowledge:

\begin{itemize}
	\item Calculus, Linear Algebra, Probability Theory. These are the fundamental mathematical knowledge that most readers should have learned during undergraduate study. You should at least understand what a matrix and a vector are, and what it means by doing differentiation and integration. For more advanced mathematical knowledge required, we will introduce in this book as we proceed.
	\item Basic C++ Programming. As we will be using C++ as our major programming language, it is recommended that the readers are at least familiar with its basic concepts and syntax. For example, you should know what a class is, how to use the C++ standard library, how to use template classes, etc. We will try our best to avoid using tricks, but in certain situations, we really can not avert. In addition, we will adopt some of the C++11 standard, but don't worry, they will be explained as they appear.
	
	\item Linux Basics. Our development environment is Linux instead of Windows, and we will only provide source code for Linux. \textbf{We believe that mastering Linux is an essential skill for SLAM researchers, and please take it to begin. After going through the contents of this book, we believe you will agree with us} \footnote{Linux is not that popular in China as our computer science education starts very lately around the 1990s.}. In Linux, the configuration of related libraries is so convenient, and you will gradually appreciate the benefit of mastering it. If you have never used a Linux system, it will be beneficial if you can find some Linux learning materials and spend some time reading them (to master Linux basics, the first few chapters of an introductory book should be sufficient). We do not ask readers to have superb Linux operating skills, but we do hope readers at least know how to fire a terminal and enter a code directory. There are some self-test questions on Linux at the end of this chapter. If you have answers to them, you shouldn't have much problem understanding the code in this book.
\end{itemize}


Readers interested in SLAM but do not have the above-mentioned knowledge may find it difficult to proceed with this book. If you do not understand the basics of C++, you can read some introductory books such as \textit{C ++ Primer Plus}. If you do not have the relevant math knowledge, we also suggest that you read some relevant math textbooks first. Nevertheless, we think that most readers who have completed undergraduate study should already have the necessary mathematical arsenal. Regarding the code, we recommend that you spend time typing them by yourself and tweaking the parameters to see how they affect outputs. This will be very helpful.

This book can be used as a textbook for SLAM-related courses, but also suitable as extra-curricular self-study materials.

\section{Style}

This book covers both mathematical theory and programming implementation. Therefore, for the convenience of reading, we will be using different layouts to distinguish different contents.

\begin{enumerate}
	\item Mathematical formulas will be listed separately, and important formulas will be assigned with an equation number on the right end of the line, for example:
	
	\begin{equation}
	\mathbf{y} =\mathbf{A}\mathbf{x}.
	\end{equation}
	
	Italics are used for scalars, e.g., $a$. Bold symbols are used for vectors and matrices, e.g.$\mathbf{a}, \mathbf{A}$. Hollow bold represents special sets, e.g.\ real number $\mathbb{R}$ and integer set $\mathbb{Z}$. Gothic is used for Lie Algebra, e.g. $\mathfrak{se}(3)$.
	
	\item Source code will be framed into boxes, using smaller font size, with line numbers on the left. If a code block is long, the box may continue to the next page:
	\begin{lstlisting}[language=C++]
	#include <iostream>
	using namespace std;
	
	int main (int argc, char** argv) {
		cout << "Hello" << endl;
		return 0;
	}
	\end{lstlisting}
	
	\item When the code block is too long or contains repeated parts with previously listed code, it is not appropriate to be listed entirely. We will only give \textbf{important snippets} and mark it with ``Snippet''. Therefore, we strongly recommend that readers download all the source code on GitHub and complete the exercises to better understand the book.
	
	\item Due to typographical reasons, the code shown in the book may be slightly different from the code hosted on GitHub. In that case please use the code on GitHub.
	
	\item For each of the libraries we use, it will be explained in detail when first appearing, but not repeated in the follow-up. Therefore, it is recommended that readers read this book in order.
	
	\item An abstract will be presented at the beginning of each lecture. A summary and some exercises will be given at the end. The cited references are listed at the end of the book.

	\item The chapters with an asterisk mark in front are optional readings, and readers can read them according to their interest. Skipping them will not hinder the understanding of subsequent chapters.
	
	\item Important contents will be marked in \textbf{bold} or \emph{italic}, as we are already accustomed to.

	\item Most of the experiments we designed are demonstrative. Understanding them does not mean that you are already familiar with the entire library. So we recommend that you spend time on yourselves in further exploring the important libraries frequently used in the book.

	\item The book's exercises and optional readings may require you to search for additional materials, so you need to learn to use search engines.
\end{enumerate}

\section{Acknowledgments}
The online English version of this book is currently publicly available and open-source. % \textcolor{red}{The Chinese version}

\section{Exercises (self-test questions)}
\begin{enumerate}
	\item There is a linear equation $\mathbf{Ax}=\mathbf{b}$, if $\mathbf{A}$ and $\mathbf{b}$ are known, how to solve for $\mathbf{x}$? What are the requirements for $\mathbf{A}$ and $\mathbf{b}$ if we want a unique $\mathbf{x}$? (Hint: check the rank of $\mathbf{A}$ and $\mathbf{b}$).
	
	\item What is a Gaussian distribution? What does it look like in a one-dimensional case? How about in a high-dimensional case?
	
	\item Do you know what a \textbf{class} is in C++? Do you know STL? Have you ever used them?
	
	\item How do you write a C++ program? (It's completely fine if your answer is ``using Visual C++ 6.0'' \footnote{As I know many of our undergraduate students are still using this VC++ 6.0 in the university. }. As long as you have C++ or C programming experience, you are in good hands).
	
	\item Do you know the C++11 standard? Which new features have you heard of or used? Are you familiar with any other standard?
	
	\item Do you know Linux? Have you used at least one flavor (not including Android), such as Ubuntu?
	
	\item What is the directory structure of Linux? What basic commands do you know? (e.g.\ ls, cat, etc.)
	
	\item How to install a software in Ubuntu (without using the Software Center)? What directories are software usually installed under? If you only know the fuzzy name of a software (for example, you want to install a library with the word ``eigen'' in its name), how would you do it?
	
	\item *Spend an hour learning Vim, you will be using it sooner or later. You can type ``vimtutor'' into a terminal and read through its contents. We do not require you to operate it very skillfully, as long as you can use it to edit the code in the process of learning this book. Do not waste time on its plugins, for now, do not try to turn Vim into an IDE, we will only use it for text editing in this book.
	
\end{enumerate}
