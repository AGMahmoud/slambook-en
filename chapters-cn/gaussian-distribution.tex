% !Mode:: "TeX:UTF-8"
\chapter{高斯分布的性质}
这里总结一下常见的高斯分布的性质,它在本书的很多地方都会用到。
\section{高斯分布}
如果一个随机变量$x$服从高斯分布$N(\mu, \sigma^2)$,那么它的概率密度函数为
\begin{equation}
p\left( x \right) = \frac{1}{{\sqrt {2\uppi } \sigma }}\exp \left( { - \frac{1}{2}\frac{{{{\left( {x - \mu } \right)}^2}}}{{{\sigma ^2}}}} \right).
\end{equation}

其高维形式为
\begin{equation}
p\left( x \right) = \frac{1}{{\sqrt {(2\uppi)^N  \det \left( \bm{\Sigma } \right) }}}\exp \left( { - \frac{1}{2}{{\left( {\bm{x} - \bm{\mu} } \right)}^\mathrm{T}}{\bm{\Sigma} ^{ - 1}}\left( {\bm{x} - \bm{\mu} } \right)} \right).
\end{equation}

\section{高斯分布的运算}
\subsection{线性运算}
设两个独立的高斯分布:
\[
\bm{x} \sim N( \bm{\mu}_x, \bm{\Sigma}_{xx} ), \quad \bm{y} \sim N( \bm{\mu}_y, \bm{\Sigma}_{yy} ),
\]
那么,它们的和仍是高斯分布:
\begin{equation}
\bm{x}+\bm{y} \sim N( \bm{\mu}_x + \bm{\mu}_y, \bm{\Sigma}_{xx} + \bm{\Sigma}_{yy}).
\end{equation}

如果以常数$a$乘以$\bm{x}$,那么$a \bm{x}$满足:
\begin{equation}
a\bm{x} \sim N( a \bm{\mu}_x, a^2 \bm{\Sigma}_{xx}).
\end{equation}

如果取$\bm{y} = \bm{A} \bm{x}$,那么$\bm{y}$满足:
\begin{equation}
\bm{y} \sim N( \bm{A} \bm{\mu}_x, \bm{A} \bm{\Sigma}_{xx} \bm{A}^\mathrm{T}).
\end{equation}

\subsection{乘积}
设两个高斯分布的乘积满足$p\left( \bm{xy} \right) = N\left( {\bm{\mu} ,\bm{\Sigma}} \right)$,那么:
\begin{equation}
\begin{array}{l}
{\bm{\Sigma}^{-1}} = \bm{\Sigma}_{xx}^{-1} + \bm{\Sigma}_{yy}^{-1} \\
\bm{\Sigma}^{-1} \bm{\mu} = \bm{\Sigma}_{xx}^{-1}{\bm{\mu}_x} + \bm{\Sigma}_{yy}^{-1}{\bm{\mu}_y}.
\end{array}
\end{equation}

该公式可以推广到任意多个高斯分布之乘积。

\subsection{复合运算}
同样考虑$\bm{x}$和$\bm{y}$,若其不独立,则其复合分布为
\begin{equation}
p(\bm{x}, \bm{y}) = N\left( {\left[ {\begin{array}{*{20}{c}}
		{{\bm{\mu}_x}}\\
		{{\bm{\mu}_y}}
		\end{array}} \right],\left[ {\begin{array}{*{20}{c}}
		{{\bm{\Sigma}_{xx}}}&{{\bm{\Sigma}_{xy}}}\\
		{{\bm{\Sigma}_{yx}}}&{{\bm{\Sigma}_{yy}}}
		\end{array}} \right]} \right).
\end{equation}

由条件分布展开式$p\left( {\bm{x}, \bm{y}} \right) = p\left( {\bm{x}|\bm{y}} \right)p\left( \bm{y} \right)$可以推出,条件概率$p(\bm{x}|\bm{y})$满足:
\begin{equation}
p\left( {\bm{x} | \bm{y} } \right) = N\left( {{\bm{\mu}_x} + {\bm{\Sigma}_{xy}} \bm{\Sigma}_{yy}^{ - 1} \left( {\bm{y} - {\bm{\mu}_y}} \right),{\bm{\Sigma}_{xx}} - {\bm{\Sigma}_{xy}} \bm{\Sigma}_{yy}^{ - 1}{\bm{\Sigma}_{yx}}} \right).
\end{equation}

\section{复合的例子}
\label{sec:gauss-example}
下面举一个和卡尔曼滤波器相关的例子。考虑随机变量$\bm{x} \sim N( \bm{\mu}_x, \bm{\Sigma}_{xx})$,另一变量$\bm{y}$满足:
\begin{equation}
\bm{y} = \bm{Ax} + \bm{b} + \bm{w}
\end{equation}

其中$\bm{A}, \bm{b}$为线性变量的系数矩阵和偏移量,$\bm{w}$为噪声项,为零均值的高斯分布:$\bm{w} \sim N(\bm{0}, \bm{R})$。

我们来看$\bm{y}$的分布。根据前面的介绍,可以推出:
\begin{equation}
p\left( \bm{y} \right) = N\left( {\bm{A}{\bm{\mu}_x} + \bm{b}, \bm{R} + \bm{A} {\bm{\Sigma}_{xx}}{\bm{A}^\mathrm{T}}} \right).
\end{equation}

这为卡尔曼滤波器的预测部分提供了理论基础。